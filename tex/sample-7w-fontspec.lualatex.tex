\documentclass{ltjsarticle}

\usepackage{luatexja-fontspec}
\setmainjfont[
  FontFace = {el}{\shapedefault}{ *-ExtraLight },
  FontFace = {sl}{\shapedefault}{ *-Light },
  FontFace = {x}{\shapedefault}{ *-Medium },
  FontFace = {sb}{\shapedefault}{ *-SemiBold },
  FontFace = {eb}{\shapedefault}{ *-Heavy }% eb の Bold を Heavy で上書き
]{HaranoAjiMincho}
\setsansjfont[
  FontFace = {el}{\shapedefault}{ *-ExtraLight },
  FontFace = {l}{\shapedefault}{ *-Light },% l の Regular を Light で上書き
  FontFace = {sl}{\shapedefault}{ *-Normal },
  FontFace = {x}{\shapedefault}{ *-Regular },
  FontFace = {sb}{\shapedefault}{ *-Medium },
]{HaranoAjiGothic}

\begin{document}

\section{Lua\LaTeX +fontspecで明朝・ゴシック各7ウェイトを使う}

CTAN haranoajiおよび
haranoaji-extraパッケージ収録分のフォントを使用。

\texttt{HaranoAjiMincho.fontspec},
\texttt{HaranoAjiGothic.fontspec}が必要。

\vspace{\baselineskip}

\rmfamily
\mcfamily
\fontseries{el}\selectfont
明朝el (Extra Light)原ノ味明朝ExtraLight

\fontseries{l}\selectfont
明朝l (Light)原ノ味明朝Light

\fontseries{sl}\selectfont
明朝sl (Semi Light)原ノ味明朝Lightで代替
\footnote{ゴシックと対称になるように割り当てた。}

\mdseries
明朝m (Medium)原ノ味明朝Regular

\fontseries{x}\selectfont
明朝x (Expanded%
\footnote{本来はウェイトではなく幅のシリーズ名だが、
  mとbの間が足りないので使用。})原ノ味明朝Medium

\fontseries{sb}\selectfont
明朝sb (Semi Bold)原ノ味明朝SemiBold

\bfseries
明朝b (Bold)原ノ味明朝Bold

\fontseries{eb}\selectfont
明朝eb (Extra Bold)原ノ味明朝Heavy

\sffamily
\gtfamily
\fontseries{el}\selectfont
ゴシックel (Extra Light)原ノ味角ゴシックExtraLight

\fontseries{l}\selectfont
ゴシックl (Light)原ノ味角ゴシックLight

\fontseries{sl}\selectfont
ゴシックsl (Semi Light)原ノ味角ゴシックNormal

\mdseries
ゴシックm (Medium)原ノ味角ゴシックRegular

\fontseries{x}\selectfont
ゴシックx (Expanded)原ノ味角ゴシックRegularで代替
\footnote{明朝と対称になるよう割り当てた。}

\fontseries{sb}\selectfont
ゴシックsb (Semi Bold)原ノ味角ゴシックMedium

\bfseries
ゴシックb (Bold)原ノ味角ゴシックBold

\fontseries{eb}\selectfont
ゴシックeb (Extra Bold)原ノ味角ゴシックHeavy

\end{document}
